\documentclass[12pt]{article}
\usepackage{pmmeta}
\pmcanonicalname{JacobisIdentityForvarthetaFunctions}
\pmcreated{2013-03-22 14:46:45}
\pmmodified{2013-03-22 14:46:45}
\pmowner{rspuzio}{6075}
\pmmodifier{rspuzio}{6075}
\pmtitle{Jacobi's identity for $\vartheta$ functions}
\pmrecord{5}{36427}
\pmprivacy{1}
\pmauthor{rspuzio}{6075}
\pmtype{Theorem}
\pmcomment{trigger rebuild}
\pmclassification{msc}{33E05}

% this is the default PlanetMath preamble.  as your knowledge
% of TeX increases, you will probably want to edit this, but
% it should be fine as is for beginners.

% almost certainly you want these
\usepackage{amssymb}
\usepackage{amsmath}
\usepackage{amsfonts}

% used for TeXing text within eps files
%\usepackage{psfrag}
% need this for including graphics (\includegraphics)
%\usepackage{graphicx}
% for neatly defining theorems and propositions
%\usepackage{amsthm}
% making logically defined graphics
%%%\usepackage{xypic}

% there are many more packages, add them here as you need them

% define commands here
\begin{document}
Jacobi's identities describe how theta functions transform under replacing the period with the negative of its reciprocal.  Together with the quasiperiodicity relations, they describe the transformations of theta functions under the modular group.
$$\theta_1 (z \mid -1/\tau) = -i (-i \tau)^{1/2} e^{i \tau z^2 \over \pi} \theta_1 (\tau z \mid \tau)$$
$$\theta_2 (z \mid -1/\tau) = (-i \tau)^{1/2} e^{i \tau z^2 \over \pi} \theta_4 (\tau z \mid \tau)$$
$$\theta_3 (z \mid -1/\tau) = (-i \tau)^{1/2} e^{i \tau z^2 \over \pi} \theta_3 (\tau z \mid \tau)$$
$$\theta_4 (z \mid -1/\tau) = (-i \tau)^{1/2} e^{i \tau z^2 \over \pi} \theta_2 (\tau z \mid \tau)$$
%%%%%
%%%%%
\end{document}
