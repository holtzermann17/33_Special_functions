\documentclass[12pt]{article}
\usepackage{pmmeta}
\pmcanonicalname{ApplyingGeneratingFunction}
\pmcreated{2013-03-22 19:06:58}
\pmmodified{2013-03-22 19:06:58}
\pmowner{pahio}{2872}
\pmmodifier{pahio}{2872}
\pmtitle{applying generating function}
\pmrecord{11}{42010}
\pmprivacy{1}
\pmauthor{pahio}{2872}
\pmtype{Example}
\pmcomment{trigger rebuild}
\pmclassification{msc}{33B99}
\pmclassification{msc}{33C45}
\pmclassification{msc}{26C05}
\pmclassification{msc}{26A42}
\pmrelated{AreaUnderGaussianCurve}

\endmetadata

% this is the default PlanetMath preamble.  as your knowledge
% of TeX increases, you will probably want to edit this, but
% it should be fine as is for beginners.

% almost certainly you want these
\usepackage{amssymb}
\usepackage{amsmath}
\usepackage{amsfonts}

% used for TeXing text within eps files
%\usepackage{psfrag}
% need this for including graphics (\includegraphics)
%\usepackage{graphicx}
% for neatly defining theorems and propositions
 \usepackage{amsthm}
% making logically defined graphics
%%%\usepackage{xypic}

% there are many more packages, add them here as you need them

% define commands here

\theoremstyle{definition}
\newtheorem*{thmplain}{Theorem}

\begin{document}
\PMlinkescapeword{formula}
The generating function of a function sequence carries information common to the members of the sequence.\, It may be utilised for deriving various properties, such as recurrence relations, orthogonality properties etc.\, We take as example
\begin{align}
e^{2zt-t^2} \;=\; \sum_{n=0}^\infty\frac{H_n(z)}{n!}t^n,
\end{align}
the \PMlinkid{generating function of the of Hermite polynomials}{11980}, and derive from it a recurrence relation and the \PMlinkname{orthonormality}{Orthonormal} formula.\\

1.\, First we form the partial derivative with respect to $t$ of both \PMlinkescapetext{sides} of (1):
$$(2z\!-\!2t)e^{2zt-t^2} \;=\; \sum_{m=1}^\infty\frac{H_m(z)}{(m\!-\!1)!}t^{m-1}$$
Here we substitute (1) to the left hand side and rewrite the right hand side, getting
$$\sum_{n=0}^\infty\frac{2zH_n(z)}{n!}t^n 
-\sum_{n=1}^\infty\frac{2H_{n-1}(z)}{(n\!-\!1)!}t^n
 \;=\; \sum_{n=0}^\infty\frac{H_{n+1}(z)}{n!}t^n,$$
where we can compare the coefficients of $t^n$:
$$\frac{2zH_n}{n!}-\frac{2H_{n-1}}{(n\!-\!1)!} \;=\; \frac{H_{n+1}}{n!} \qquad (n = 1,\,2,\,\ldots)$$
Thus we have gotten the recurrence relation
\begin{align}
H_{n+1}(z) \;=\; 2zH_n(z)-2nH_{n-1}(z) \qquad (n = 1,\,2,\,\ldots).
\end{align}
Differentiating (1) partially with respect to $z$ enables respectively to find a formula expressing the derivative 
$H_n'(z)$ via the \PMlinkescapetext{polynomials} themselves.\\

2.\, We copy the equation (1) twice in the forms
$$\sum_{n=0}^\infty\frac{H_n(x)}{n!}t^n \;=\; e^{2xt-t^2}, \quad 
  \sum_{n=0}^\infty\frac{H_n(x)}{n!}u^n \;=\; e^{2xu-u^2},$$
multiply these with each other and by $e^{-x^2}$ and then integrate the obtained equation termwise over 
$\mathbb{R}$:
\begin{align*}
\sum_{m=0}^\infty\sum_{n=0}^\infty\left(\int_{-\infty}^\infty\!e^{-x^2}H_m(x)H_n(x)\,dx\right)\frac{t^mu^n}{m!n!} 
\;=\; & \int_{-\infty}^\infty\!e^{-x^2}e^{2xt-t^2}e^{2xu-u^2}\,dx \\
\;=\; & \int_{-\infty}^\infty\!e^{2x(t+u)-t^2-u^2-x^2}\,dx \\
\;=\; & \int_{-\infty}^\infty\!e^{-[(t+u)^2-2(t+u)x+x^2]+2tu}\,dx \\
\;=\; & e^{2tu}\!\int_{-\infty}^\infty\!e^{-[x-(t+u)]^2}\,dx \\
\;=\; & e^{2tu}\!\int_{-\infty}^\infty\!e^{-y^2}\,dy \\
\;=\; & e^{2tu}\sqrt{\pi} \\
\;=\; & \sum_{\j=0}^\infty\sqrt{\pi}\frac{2^jt^ju^j}{j!} \\
\;=\; & \sum_{m=0}^\infty\sum_{n=0}^\infty\left(\frac{\sqrt{\pi}}{n!}\cdot2^n\delta_{mn}\right)t^mu^n
\end{align*}
Thus we can infer that
$$\frac{\int_{-\infty}^\infty\!e^{-x^2}H_m(x)H_n(x)\,dx}{m!n!} \;=\; \frac{\sqrt{\pi}}{n!}\cdot2^n\delta_{mn},$$
which implies the orthonormality relation
\begin{align}
\int_{-\infty}^\infty\!e^{-x^2}H_m(x)H_n(x)\,dx \;=\; 2^mm!\,\delta_{mn}\sqrt{\pi}.
\end{align}
Cf. Hermite polynomials.

%%%%%
%%%%%
\end{document}
