\documentclass[12pt]{article}
\usepackage{pmmeta}
\pmcanonicalname{ConvergenceOfTheSequence11nn}
\pmcreated{2013-03-22 17:43:26}
\pmmodified{2013-03-22 17:43:26}
\pmowner{kfgauss70}{18761}
\pmmodifier{kfgauss70}{18761}
\pmtitle{convergence of the sequence (1+1/n)^n}
\pmrecord{7}{40170}
\pmprivacy{1}
\pmauthor{kfgauss70}{18761}
\pmtype{Theorem}
\pmcomment{trigger rebuild}
\pmclassification{msc}{33B99}
\pmrelated{NondecreasingSequenceWithUpperBound}

\endmetadata

% this is the default PlanetMath preamble.  as your knowledge
% of TeX increases, you will probably want to edit this, but
% it should be fine as is for beginners.

% almost certainly you want these
\usepackage{amssymb}
\usepackage{amsmath}
\usepackage{amsfonts}

% used for TeXing text within eps files
%\usepackage{psfrag}
% need this for including graphics (\includegraphics)
%\usepackage{graphicx}
% for neatly defining theorems and propositions
\usepackage{amsthm}
% making logically defined graphics
%%%\usepackage{xypic}

% there are many more packages, add them here as you need them

% define commands here
\newtheorem{theorem}{Theorem}
\begin{document}
\begin{theorem}
The following sequence:
\begin{equation}
a_n=\left(1+\frac{1}{n}\right)^n
\label{eqn001}
\end{equation}
is convergent.
\end{theorem}
\begin{proof}
The proof will be given by demonstrating that the sequence (\ref{eqn001}) is:
\begin{enumerate}
\item monotonic (increasing), that is $a_{n}<a_{n+1}$
\item bounded above, that is $\forall n\in\mathbb{N}, a_n<M$ for some $M>0$
\end{enumerate}
In order to prove part 1, consider the binomial expansion for $a_n$:
$$
a_n=\sum_{k=0}^{n}\binom{n}{k}\frac{1}{n^k}=\sum_{k=0}^{n}\frac{1}{k!}\frac{n}{n}\frac{n-1}{n}\ldots\frac{n-(k-1)}{n}=\sum_{k=0}^{n}\frac{1}{k!}\left(1-\frac{1}{n}\right)\ldots\left(1-\frac{k-1}{n}\right).
$$
Since $\forall i\in\{1,2\ldots (k-1)\}:(1-\frac{i}{n})<(1-\frac{i}{n+1})$, and since the sum $a_{n+1}$ has one term more than $a_n$, it is demonstrated that the sequence (\ref{eqn001}) is monotonic.\\
In order to prove part 2, consider again the binomial expansion:
$$
a_n=1+\frac{n}{n}+\frac{1}{2!}\frac{n(n-1)}{n^2}+\frac{1}{3!}\frac{n(n-1)(n-2)}{n^3}+\ldots+\frac{1}{n!}\frac{n(n-1)\ldots(n-n+1)}{n^n}.
$$
Since $\forall k\in\{2,3\ldots n\}:\frac{1}{k!}<\frac{1}{2^{k-1}}$ and $\frac{n(n-1)\ldots(n-(k-1))}{n^k}<1$: 
$$
a_n<1+\left(1+\frac{1}{2}+\frac{1}{2\times2}+\ldots+\frac{1}{2^{n-1}}\right)<1+\left(\frac{1-\frac{1}{2^n}}{1-\frac{1}{2}}\right)<3-\frac{1}{2^{n-1}}<3
$$
where the formula giving the sum of the geometric progression with ratio $1/2$ has been used.
\end{proof}
\noindent In conclusion, we can say that the sequence (\ref{eqn001}) is convergent and its limit corresponds to the supremum of the set $\{a_n\}\subset\left[2,3\right)$, denoted by $e$, that is:
$$
\lim_{n\to\infty}\left(1+\frac{1}{n}\right)^n=\sup_{n\in\mathbb{N}}\left\{\left(1+\frac{1}{n}\right)^n\right\}\triangleq e,
$$
which is the definition of the Napier's constant.
%%%%%
%%%%%
\end{document}
