\documentclass[12pt]{article}
\usepackage{pmmeta}
\pmcanonicalname{GlobalCharacterizationOfHypergeometricFunction}
\pmcreated{2014-12-31 15:15:16}
\pmmodified{2014-12-31 15:15:16}
\pmowner{rspuzio}{6075}
\pmmodifier{rspuzio}{6075}
\pmtitle{global characterization of hypergeometric function}
\pmrecord{6}{40021}
\pmprivacy{1}
\pmauthor{rspuzio}{6075}
\pmtype{Definition}
\pmcomment{trigger rebuild}
\pmclassification{msc}{33C05}

% this is the default PlanetMath preamble.  as your knowledge
% of TeX increases, you will probably want to edit this, but
% it should be fine as is for beginners.

% almost certainly you want these
\usepackage{amssymb}
\usepackage{amsmath}
\usepackage{amsfonts}

% used for TeXing text within eps files
%\usepackage{psfrag}
% need this for including graphics (\includegraphics)
%\usepackage{graphicx}
% for neatly defining theorems and propositions
%\usepackage{amsthm}
% making logically defined graphics
%%%\usepackage{xypic}

% there are many more packages, add them here as you need them

% define commands here

\begin{document}
Riemann noted that the hypergeometric function can be characterized
by its global properties, without reference to power series, differential
equations, or any other sort of explicit expression.  His characterization
is conveniently restated in terms of sheaves:

Suppose that we have a sheaf of holomorphic functions over $\mathbb{C} 
\setminus \{0,1\}$ which satisfy the following properties:
\begin{itemize}
\item It is closed under analytic continuation.
\item It is closed under taking linear combinations.
\item The space of function elements over any open set is two dimensional.
\item There exists a neighborhood $D_0$ such that $0 \in D)$, holomorphic 
functions $\phi_0, \psi_0$ defined on $D_0$, and complex numbers $\alpha_0,
\beta_0$ such that, for an open set of $d_0$ not containing $0$, it happens that 
$z \mapsto z^{\alpha_0} \phi(z)$ and $z \mapsto z^{\beta_0} \psi(z)$ belong to 
our sheaf.
\end{itemize}
Then the sheaf consists of solutions to a hypergeometric equation, hence
the function elements are hypergeometric functions.
%%%%%
%%%%%
\end{document}
