\documentclass[12pt]{article}
\usepackage{pmmeta}
\pmcanonicalname{HyperbolicIdentities}
\pmcreated{2013-03-22 17:50:42}
\pmmodified{2013-03-22 17:50:42}
\pmowner{Wkbj79}{1863}
\pmmodifier{Wkbj79}{1863}
\pmtitle{hyperbolic identities}
\pmrecord{11}{40316}
\pmprivacy{1}
\pmauthor{Wkbj79}{1863}
\pmtype{Topic}
\pmcomment{trigger rebuild}
\pmclassification{msc}{33B10}
\pmclassification{msc}{26A09}
\pmsynonym{hyperbolic formulas}{HyperbolicIdentities}
\pmsynonym{hyperbolic formulae}{HyperbolicIdentities}
\pmrelated{GoniometricFormulae}
\pmrelated{AdditionAndSubtractionFormulasForHyperbolicFunctions}
\pmrelated{ExamplesOfPeriodicFunctions}
\pmrelated{TaylorSeriesOfHyperbolicFunctions}

\usepackage{amssymb}
\usepackage{amsmath}
\usepackage{amsfonts}
\usepackage{pstricks}
\usepackage{psfrag}
\usepackage{graphicx}
\usepackage{amsthm}
%%\usepackage{xypic}

\newcommand{\sech}{\operatorname{sech}}
\newcommand{\csch}{\operatorname{csch}}

\newcommand{\ds}{\displaystyle}
\begin{document}
\PMlinkescapeword{formulas}

There are many formulas involving hyperbolic functions, many of which are \PMlinkescapetext{similar} to formulas for trigonometric functions.  Below is a list of some of these formulas (usually for real arguments).

\begin{enumerate}
\item Hyperbolic version of Pythagorean identities

\begin{itemize}
\item $\cosh^2 x-\sinh^2 x=1$
\item $1-\tanh^2 x=\sech^2 x$
\item $\coth^2 x-1=\csch^2 x$
\end{itemize}

\item Fractional identities

\begin{itemize}
\item $\ds \tanh x=\frac{\sinh x}{\cosh x}$
\item $\ds \coth x=\frac{\cosh x}{\sinh x}$
\item $\ds \coth x=\frac{1}{\tanh x}$
\item $\ds \tanh x=\frac{1}{\coth x}$
\item $\ds \csch x=\frac{1}{\sinh x}$
\item $\ds \sech x=\frac{1}{\cosh x}$
\end{itemize}

\item Hyperbolic functions of a purely imaginary number

\begin{itemize}
\item $\sinh(ix)=i\sin x$
\item $\cosh(ix)=\cos x$
\item $\tanh(ix)=i\tan x$
\item $\cosh(ix)=i\cot x$
\item $\csch(ix)=i\csc x$
\item $\sech(ix)=\sec x$
\end{itemize}

\item \PMlinkname{Addition formulas and subtraction formulas}{AdditionAndSubtractionFormulasForHyperbolicFunctions}

\begin{itemize}
\item $\sinh(x\pm y)=\sinh x\cosh y\pm\cosh x\sinh y$
\item $\cosh(x\pm y)=\cosh x\cosh y\pm\sinh x\sinh y$
\item $\ds \tanh(x\pm y)=\frac{\tanh x\pm\tanh y}{1\pm\tanh x\tanh y}$
\end{itemize}

\item Formulas for hyperbolic functions of a complex number

\begin{itemize}
\item $\sinh(x+iy)=\sinh x\cos y+i\cosh x\sin y$
\item $\cosh(x+iy)=\cosh x\cos y+i\sinh x\sin y$
\item $\ds \tanh(x+iy)=\frac{\tanh x+i\tan y}{1+i\tanh x\tan y}$
\end{itemize}

\item Opposite formulas

\begin{itemize}
\item $\sinh(-x)=-\sinh x$
\item $\cosh(-x)=\cosh x$
\item $\tanh(-x)=-\tanh x$
\end{itemize}

\item Double argument formulas

\begin{itemize}
\item $\sinh(2x)=2\sinh x\cosh x$
\item $\cosh(2x)=\cosh^2 x+\sinh^2 x=2\cosh^2 x-1=1+2\sinh^2 x$
\item $\ds \tanh(2x)=\frac{2\tanh x}{1+\tanh^2 x}$
\end{itemize}

\item \PMlinkname{Periodicity}{Periodic} formulas

\begin{itemize}
\item $\sinh(z+2\pi i) = \sinh{z}$
\item $\cosh(z+2\pi i) = \cosh{z}$
\item $\tanh(z+\pi i) = \tanh{z}$
\end{itemize}

\PMlinkname{Cf}{Cf}. the periodicity of exponential function.

\item \PMlinkname{Exponential formulas}{ExponentialFunction}

\begin{itemize}
\item $\ds \cosh x=\frac{e^x+e^{-x}}{2}$
\item $\ds \sinh x=\frac{e^x-e^{-x}}{2}$
\item $\ds \tanh x=\frac{e^x-e^{-x}}{e^x+e^{-x}}$
\item $e^x=\cosh x+\sinh x$
\item $e^{-x}=\cosh x-\sinh x$
\end{itemize}

Note that the first three formulas given in this \PMlinkescapetext{section} are definitions.

\end{enumerate}
%%%%%
%%%%%
\end{document}
