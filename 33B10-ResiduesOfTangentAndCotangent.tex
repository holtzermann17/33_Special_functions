\documentclass[12pt]{article}
\usepackage{pmmeta}
\pmcanonicalname{ResiduesOfTangentAndCotangent}
\pmcreated{2013-03-22 18:57:35}
\pmmodified{2013-03-22 18:57:35}
\pmowner{pahio}{2872}
\pmmodifier{pahio}{2872}
\pmtitle{residues of tangent and cotangent}
\pmrecord{6}{41817}
\pmprivacy{1}
\pmauthor{pahio}{2872}
\pmtype{Example}
\pmcomment{trigger rebuild}
\pmclassification{msc}{33B10}
\pmclassification{msc}{30D10}
\pmclassification{msc}{30A99}
\pmrelated{Residue}
\pmrelated{TechniqueForComputingResidues}
\pmrelated{ResiduesOfGammaFunction}

\endmetadata

% this is the default PlanetMath preamble.  as your knowledge
% of TeX increases, you will probably want to edit this, but
% it should be fine as is for beginners.

% almost certainly you want these
\usepackage{amssymb}
\usepackage{amsmath}
\usepackage{amsfonts}

% used for TeXing text within eps files
%\usepackage{psfrag}
% need this for including graphics (\includegraphics)
%\usepackage{graphicx}
% for neatly defining theorems and propositions
 \usepackage{amsthm}
% making logically defined graphics
%%%\usepackage{xypic}

% there are many more packages, add them here as you need them

% define commands here

\theoremstyle{definition}
\newtheorem*{thmplain}{Theorem}

\begin{document}
We will determine the residues of the tangent and the cotangent at their poles, which by the \PMlinkid{parent entry}{9074} are \PMlinkname{simple}{SimplePole}.

By the rule in the entry coefficients of Laurent series, in a simple pole \,$z = a$\, of $f$ one has
$$\mbox{Res}(f;\, a) \;=\; \lim_{z \to a}(z\!-\!a)f(z).$$

\begin{itemize}

\item We get first
\begin{align}
\mbox{Res}(\cot;\,0) \;=\; \lim_{z \to 0}z\cot{z} \;=\; \lim_{z \to 0}\frac{\cos{z}}{\frac{\sin{z}}{z}} 
\;=\; \frac{1}{1} \;=\;1.
\end{align}

\item All the poles of cotangent are \,$n\pi$\, with\, $n \in \mathbb{Z}$.\, Since $\pi$ is the period of cotangent, we could infer that the residues in all poles are the same as (1).\, We may also calculate (with the change of variable 
\,$z\!-\!n\pi = w$) directly
$$\mbox{Res}(\cot;\,n\pi) \;=\; \lim_{z \to n\pi}(z\!-\!n\pi)\cot{z} 
\;=\; \lim_{w \to 0}w\cot(w\!+\!n\pi) \;=\; \lim_{w \to 0}w\cot{w} \;=\; 1.$$

\item In the \PMlinkname{parent entry}{ComplexTangentAndCotangent}, the complement formula for the tangent function is derived.\, Using it, we can find the residues of tangent at its poles $\displaystyle\frac{\pi}{2}+n\pi$, which are \PMlinkescapetext{simple}.\, For example,
$$\mbox{Res}(\tan;\,\frac{\pi}{2}) \;=\; 
\lim_{z \to \frac{\pi}{2}}\left(z\!-\!\frac{\pi}{2}\right)\cot\left(\frac{\pi}{2}\!-\!z\right)
\;=\; \lim_{w \to 0}w\cot(-w) \;=\; -\mbox{Res}(\cot;\,0) \;=\; -1.$$
Similarly as above, the residues in other poles are $-1$.

\end{itemize}

Consequently, the residues of cotangent are equal to 1 and the residues of tangent equal to $-1$.
%%%%%
%%%%%
\end{document}
