\documentclass[12pt]{article}
\usepackage{pmmeta}
\pmcanonicalname{GeneratingFunctionOfLaguerrePolynomials}
\pmcreated{2013-03-22 19:06:51}
\pmmodified{2013-03-22 19:06:51}
\pmowner{pahio}{2872}
\pmmodifier{pahio}{2872}
\pmtitle{generating function of Laguerre polynomials}
\pmrecord{8}{42007}
\pmprivacy{1}
\pmauthor{pahio}{2872}
\pmtype{Derivation}
\pmcomment{trigger rebuild}
\pmclassification{msc}{33B99}
\pmclassification{msc}{30B10}
\pmclassification{msc}{26C05}
\pmclassification{msc}{26A09}
\pmclassification{msc}{33E30}
\pmrelated{ExampleOfFindingTheGeneratingFunction}
\pmrelated{GeneratingFunctionOfHermitePolynomials}
\pmrelated{VariantOfCauchyIntegralFormula}

% this is the default PlanetMath preamble.  as your knowledge
% of TeX increases, you will probably want to edit this, but
% it should be fine as is for beginners.

% almost certainly you want these
\usepackage{amssymb}
\usepackage{amsmath}
\usepackage{amsfonts}

% used for TeXing text within eps files
%\usepackage{psfrag}
% need this for including graphics (\includegraphics)
%\usepackage{graphicx}
% for neatly defining theorems and propositions
 \usepackage{amsthm}
% making logically defined graphics
%%%\usepackage{xypic}

% there are many more packages, add them here as you need them

% define commands here

\theoremstyle{definition}
\newtheorem*{thmplain}{Theorem}

\begin{document}
We start from the definition of Laguerre polynomials via their \PMlinkid{Rodrigues formula}{11983}
\begin{align}
L_n(z) \;:=\; e^z\frac{d^n}{dz^n}e^{-z}z^n \qquad (n \;=\; 0,\,1,\,2,\,\ldots).
\end{align}
The consequence 
\begin{align}
f^{(n)}(z) \;=\; \frac{n!}{2 \pi i} \oint_C \frac{f(\zeta)}{(\zeta-z)^{n+1}}\ d\zeta
\end{align}
of \PMlinkid{Cauchy integral formula}{1150} allows to write (1) as the complex integral
$$L_n(z) \;=\; \frac{n!}{2i\pi}\oint_C\frac{e^ze^{-\zeta}}{(\zeta\!-\!z)^{n+1}}\,d\zeta
 \;=\; \frac{n!}{2i\pi}\oint_C\frac{e^{z-\zeta}\,d\zeta}{(1\!-\!\frac{z}{\zeta})^n(\zeta\!-\!z)},$$
where $C$ is any \PMlinkescapetext{closed} contour around the point $z$ and the direction is anticlockwise.\, The \PMlinkid{substitution}{11373}
$$\zeta\!-\!z \;:=\; \frac{zt}{1\!-\!t}, \quad \zeta \;=\; \frac{z}{1\!-\!t}, 
\quad t \;=\; 1\!-\!\frac{z}{\zeta} \quad d\zeta \;=\; \frac{z\,dt}{(1\!-\!t)^2}$$
here yields
$$L_n(z) \;=\; \frac{n!}{2i\pi}\oint_{C'}\frac{e^{-\frac{zt}{1-t}}z\,dt}{(1\!-\!t)^2t^n\cdot\frac{zt}{1-t}}
\;=\; \frac{n!}{2i\pi}\oint_{C'}\frac{e^{-\frac{zt}{1-t}}\,dt}{(1\!-\!t)t^{n+1}}$$
where the contour $C'$ goes round the origin.\, Accordingly, by (2) we can infer that
$$L_n(z) \;=\; \left[\frac{d^{\,n}}{dt^n}\frac{e^{-\frac{zt}{1-t}}}{1\!-\!t}\right]_{t=0},$$
whence we have found the generating function
$$\frac{e^{-\frac{zt}{1-t}}}{1\!-\!t} \;=\; \sum_{n=0}^\infty\frac{L_n(z)}{n!}t^n$$
of the Laguerre polynomials.

%%%%%
%%%%%
\end{document}
