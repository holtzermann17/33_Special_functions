\documentclass[12pt]{article}
\usepackage{pmmeta}
\pmcanonicalname{ApplicationOfLogarithmSeries}
\pmcreated{2013-03-22 18:56:09}
\pmmodified{2013-03-22 18:56:09}
\pmowner{pahio}{2872}
\pmmodifier{pahio}{2872}
\pmtitle{application of logarithm series}
\pmrecord{12}{41790}
\pmprivacy{1}
\pmauthor{pahio}{2872}
\pmtype{Application}
\pmcomment{trigger rebuild}
\pmclassification{msc}{33B10}
\pmrelated{DilogarithmFunction}
\pmrelated{ExamplesOnHowToFindTaylorSeriesFromOtherKnownSeries}
\pmrelated{SubstitutionNotation}

\endmetadata

% this is the default PlanetMath preamble.  as your knowledge
% of TeX increases, you will probably want to edit this, but
% it should be fine as is for beginners.

% almost certainly you want these
\usepackage{amssymb}
\usepackage{amsmath}
\usepackage{amsfonts}

% used for TeXing text within eps files
%\usepackage{psfrag}
% need this for including graphics (\includegraphics)
%\usepackage{graphicx}
% for neatly defining theorems and propositions
%\usepackage{amsthm}
% making logically defined graphics
%%%\usepackage{xypic}

% there are many more packages, add them here as you need them

% define commands here
\newcommand{\sijoitus}[2]%
{\operatornamewithlimits{\Big/}_{\!\!\!#1}^{\,#2}}

\begin{document}
\PMlinkescapeword{expansion}

The integrand of the improper integral
\begin{align}
I \;:=\; \int_0^1\frac{\ln(1\!+\!x)}{x}dx
\end{align}
is not defined at the lower limit 0.\, However, from the Taylor series expansion 
$$\ln(1\!+\!x) \;=\; x-\frac{x^2}{2}+\frac{x^3}{3}-\frac{x^4}{4}+-\ldots \qquad (-1 < x \leqq 1)$$
of the natural logarithm we obtain the expansion of the integrand
$$
\frac{\ln(1\!+\!x)}{x} \;=\; 1-\frac{x}{2}+\frac{x^2}{3}-\frac{x^3}{4}+-\ldots \qquad (-1 < x < 0,\;\; 0 < x \leqq 1)
$$
whence
\begin{align}
\lim_{x\to0}\frac{\ln(1\!+\!x)}{x} \;=\; 1.
\end{align}
This implies that the integrand of (1) is bounded on the interval \,$[0,\,1]$ and also continuous, if we think that (2) defines its value at\, $x = 0$.\, Accordingly, the integrand is Riemann integrable on the interval, and we can determine the improper integral by integrating termwise:
\begin{align*}
I & \;=\; \int_0^1\!\left(1-\frac{x}{2}+\frac{x^2}{3}-\frac{x^3}{4}+-\ldots\right)dx\\
  & \;=\; \sijoitus{0}{\quad1}\!\left(x-\frac{x^2}{2^2}+\frac{x^3}{3^2}-\frac{x^4}{4^2}+-\ldots\right)\\
  & \;=\; 1-\frac{1}{2^2}+\frac{1}{3^2}-\frac{1}{4^2}+-\ldots
\end{align*}
By the entry on \PMlinkname{Dirichlet eta function at 2}{ValueOfDirichletEtaFunctionAtS2}, the sum of the obtained series is\, $\eta(2) = \frac{\pi^2}{12}$.\, Thus we have the result
\begin{align}
\int_0^1\frac{\ln(1\!+\!x)}{x}dx \;=\; \frac{\pi^2}{12}.
\end{align}


Similarly, using the series
$$\ln(1\!-\!x) \;=\; -x-\frac{x^2}{2}-\frac{x^3}{3}-\frac{x^4}{4}-\ldots \qquad (-1 \leqq x < 1)$$
and the result in the entry \PMlinkname{Riemann zeta function at 2}{ValueOfTheRiemannZetaFunctionAtS2}, one can calculate that
\begin{align}
\int_0^1\frac{\ln(1\!-\!x)}{x}dx \;=\; -\frac{\pi^2}{6}.
\end{align}

%%%%%
%%%%%
\end{document}
