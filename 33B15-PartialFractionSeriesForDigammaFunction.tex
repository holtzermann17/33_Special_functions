\documentclass[12pt]{article}
\usepackage{pmmeta}
\pmcanonicalname{PartialFractionSeriesForDigammaFunction}
\pmcreated{2013-03-22 16:23:40}
\pmmodified{2013-03-22 16:23:40}
\pmowner{rm50}{10146}
\pmmodifier{rm50}{10146}
\pmtitle{partial fraction series for digamma function}
\pmrecord{6}{38540}
\pmprivacy{1}
\pmauthor{rm50}{10146}
\pmtype{Theorem}
\pmcomment{trigger rebuild}
\pmclassification{msc}{33B15}
\pmclassification{msc}{30D30}

\endmetadata

% this is the default PlanetMath preamble.  as your knowledge
% of TeX increases, you will probably want to edit this, but
% it should be fine as is for beginners.

% almost certainly you want these
\usepackage{amssymb}
\usepackage{amsmath}
\usepackage{amsfonts}

% used for TeXing text within eps files
%\usepackage{psfrag}
% need this for including graphics (\includegraphics)
%\usepackage{graphicx}
% for neatly defining theorems and propositions
%\usepackage{amsthm}
% making logically defined graphics
%%%\usepackage{xypic}

% there are many more packages, add them here as you need them

% define commands here
\newtheorem{thm}{Theorem}
\begin{document}
\begin{thm}
\[\psi (z) = - \gamma -\frac{1}{z} +
\sum_{k=1}^\infty \left( \frac{1}{k} - \frac{1}{z + k} \right)=-\gamma+\sum_{k=0}^{\infty}\left(\frac{1}{k+1}-\frac{1}{z+k}\right)\]
\end{thm}
\textbf{Proof:}
Start with
\[
\Gamma(z) = \frac{e^{-\gamma z}}{z}
\prod_{k=1}^\infty \left(1 + \frac{z}{k}\right)^{-1} e^{z/k},
\]
so
\[
\ln\Gamma(z)=-\gamma z - \ln z +\sum_{k=1}^{\infty}\left(-\ln\left(1+\frac{z}{k}\right)+\frac{z}{k}\right)\]
and thus, taking derivatives,
\[\psi(z)=-\gamma-\frac{1}{z}+\sum_{k=1}^{\infty}\left(-\frac{1/k}{1+\frac{z}{k}}+\frac{1}{k}\right)
=-\gamma-\frac{1}{z}+\sum_{k=1}^{\infty}\left(\frac{1}{k}-\frac{1}{z+k}\right)\]
The second formula follows after rearranging terms (the rearrangement is legal since we are simply exchanging adjacent terms, so partial sums remain the same).

%%%%%
%%%%%
\end{document}
