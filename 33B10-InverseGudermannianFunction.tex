\documentclass[12pt]{article}
\usepackage{pmmeta}
\pmcanonicalname{InverseGudermannianFunction}
\pmcreated{2013-03-22 19:06:28}
\pmmodified{2013-03-22 19:06:28}
\pmowner{pahio}{2872}
\pmmodifier{pahio}{2872}
\pmtitle{inverse Gudermannian function}
\pmrecord{5}{42000}
\pmprivacy{1}
\pmauthor{pahio}{2872}
\pmtype{Definition}
\pmcomment{trigger rebuild}
\pmclassification{msc}{33B10}
\pmclassification{msc}{26E05}
\pmclassification{msc}{26A48}
\pmclassification{msc}{26A09}
\pmsynonym{inverse Gudermannian}{InverseGudermannianFunction}
\pmrelated{HyperbolicFunctions}
\pmrelated{AreaFunctions}
\pmrelated{MercatorProjection}
\pmrelated{EulerNumbers2}
\pmrelated{DualityOfGudermannianAndItsInverseFunction}

\endmetadata

% this is the default PlanetMath preamble.  as your knowledge
% of TeX increases, you will probably want to edit this, but
% it should be fine as is for beginners.

% almost certainly you want these
\usepackage{amssymb}
\usepackage{amsmath}
\usepackage{amsfonts}

% used for TeXing text within eps files
%\usepackage{psfrag}
% need this for including graphics (\includegraphics)
%\usepackage{graphicx}
% for neatly defining theorems and propositions
 \usepackage{amsthm}
% making logically defined graphics
%%%\usepackage{xypic}

% there are many more packages, add them here as you need them

% define commands here

\theoremstyle{definition}
\newtheorem*{thmplain}{Theorem}

\begin{document}
Since the real Gudermannian function gd is strictly increasing and forms a bijection from $\mathbb{R}$ onto the open interval \,$(-\frac{\pi}{2},\,\frac{\pi}{2})$,\, it has an inverse function
$$\mbox{gd}^{-1}\!:\; (-\frac{\pi}{2},\,\frac{\pi}{2})\, \to\, \mathbb{R}.$$
The function $\mbox{gd}^{-1}$ is denoted also \textbf{arcgd}.\\

If\, $x = \mbox{gd}\,y$, which may be explicitly written e.g.
$$x \;=\; \arcsin(\tanh{y}),$$
one can solve this for $y$, getting first\, $\tanh{y} = \sin{x}$\, and then
$$y \;=\; \mbox{artanh}(\sin{x})$$
(see the area functions).\, Hence the \emph{inverse Gudermannian} is expressed as
\begin{align}
\mbox{gd}^{-1}(x) \;=\; \mbox{arcgd}\,x \;=\; \mbox{artanh}(\sin{x})
\end{align}
It has other \PMlinkname{equivalent}{Equivalent3} expressions, such as
\begin{align}
\mbox{gd}^{-1}(x) \;=\; \mbox{arsinh}(\tan{x}) \;=\; \frac{1}{2}\ln\frac{1+\sin{x}}{1-\sin{x}}
\;=\; \int_0^x\!\frac{dt}{\cos{t}}.
\end{align}
Thus its derivative is
\begin{align}
\frac{d}{dx}\mbox{gd}^{-1}(x) \;=\; \frac{1}{\cos{x}}.
\end{align}
Cf. the formulae (1)--(3) with the corresponding ones of gd.


%%%%%
%%%%%
\end{document}
