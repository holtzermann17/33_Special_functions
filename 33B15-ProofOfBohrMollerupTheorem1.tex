\documentclass[12pt]{article}
\usepackage{pmmeta}
\pmcanonicalname{ProofOfBohrMollerupTheorem}
\pmcreated{2013-03-22 13:18:14}
\pmmodified{2013-03-22 13:18:14}
\pmowner{lieven}{1075}
\pmmodifier{lieven}{1075}
\pmtitle{proof of Bohr-Mollerup theorem}
\pmrecord{6}{33808}
\pmprivacy{1}
\pmauthor{lieven}{1075}
\pmtype{Proof}
\pmcomment{trigger rebuild}
\pmclassification{msc}{33B15}
\pmdefines{Gau\ss's product}

\endmetadata

% this is the default PlanetMath preamble.  as your knowledge
% of TeX increases, you will probably want to edit this, but
% it should be fine as is for beginners.

% almost certainly you want these
\usepackage{amssymb}
\usepackage{amsmath}
\usepackage{amsfonts}

% used for TeXing text within eps files
%\usepackage{psfrag}
% need this for including graphics (\includegraphics)
%\usepackage{graphicx}
% for neatly defining theorems and propositions
%\usepackage{amsthm}
% making logically defined graphics
%%%\usepackage{xypic}

% there are many more packages, add them here as you need them

% define commands here
\begin{document}
We prove this theorem in two stages: first, we establish that the gamma function satisfies the given conditions and then we prove that these conditions uniquely determine a function on $(0,\infty)$.

By its definition, $\Gamma(x)$ is positive for positive $x$. Let $x,y>0$ and $0\leq \lambda \leq 1$. 

\begin{eqnarray*}
\log \Gamma(\lambda x+(1-\lambda)y) &=& \log \int_0^\infty e^{-t}t^{\lambda x+(1-\lambda)y-1}dt\\
&=&\log\int_0^\infty (e^{-t}t^{x-1})^\lambda (e^{-t}t^{y-1})^{1-\lambda}dt\\
&\leq&\log ((\int_0^\infty e^{-t}t^{x-1}dt)^\lambda(\int_0^\infty e^{-t}t^{y-1}dt)^{1-\lambda})\\
&=& \lambda\log\Gamma(x)+(1-\lambda)\log\Gamma(y)
\end{eqnarray*}

The inequality follows from H\"older's inequality, where $p=\frac{1}{\lambda}$ and $q=\frac{1}{1-\lambda}$.

This proves that $\Gamma$ is log-convex. Condition 2 follows from the definition by applying integration by parts. Condition 3 is a trivial verification from the definition.

Now we show that the 3 conditions uniquely determine a function. By condition 2, it suffices to show that the conditions uniquely determine a function on $(0,1)$.

Let $G$ be a function satisfying the 3 conditions, $0\leq x\leq 1$ and $n\in{\mathbb N}$.

$n+x=(1-x)n+x(n+1)$ and by log-convexity of $G$, $G(n+x)\leq G(n)^{1-x}G(n+1)^x=G(n)^{1-x}G(n)^xn^x=(n-1)!n^x$.

Similarly $n+1=x(n+x)+(1-x)(n+1+x)$ gives $n!\leq G(n+x)(n+x)^{1-x}$.

Combining these two we get

$$ n!(n+x)^{x-1}\leq G(n+x) \leq (n-1)!n^x $$

and by using condition 2 to express $G(n+x)$ in terms of $G(x)$ we find


$$ a_n:=\frac{n!(n+x)^{x-1}}{x(x+1)\dots(x+n-1)}\leq G(x) \leq \frac{(n-1)!n^x}{x(x+1)\dots(x+n-1)}=:b_n.$$

Now these inequalities hold for every positive integer $n$ and the terms on the left and right side have a common limit ($\lim_{n\rightarrow\infty}\frac{a_n}{b_n}=1$) so we find this determines $G$.

As a corollary we find another expression for $\Gamma$.

For $0\leq x\leq 1$,

$$ \Gamma(x)=\lim_{n\rightarrow\infty} \frac{n!n^x}{x(x+1)\dots(x+n)}.$$

In fact, this equation, called Gau\ss's product, goes for the whole complex plane minus the negative integers.
%%%%%
%%%%%
\end{document}
