\documentclass[12pt]{article}
\usepackage{pmmeta}
\pmcanonicalname{OrthogonalityOfLegendrePolynomials}
\pmcreated{2013-03-22 18:55:30}
\pmmodified{2013-03-22 18:55:30}
\pmowner{pahio}{2872}
\pmmodifier{pahio}{2872}
\pmtitle{orthogonality of Legendre polynomials}
\pmrecord{14}{41777}
\pmprivacy{1}
\pmauthor{pahio}{2872}
\pmtype{Derivation}
\pmcomment{trigger rebuild}
\pmclassification{msc}{33C45}
\pmrelated{OrthogonalPolynomials}
\pmrelated{OrthogonalityOfChebyshevPolynomials}
\pmrelated{SubstitutionNotation}

% this is the default PlanetMath preamble.  as your knowledge
% of TeX increases, you will probably want to edit this, but
% it should be fine as is for beginners.

% almost certainly you want these
\usepackage{amssymb}
\usepackage{amsmath}
\usepackage{amsfonts}

% used for TeXing text within eps files
%\usepackage{psfrag}
% need this for including graphics (\includegraphics)
%\usepackage{graphicx}
% for neatly defining theorems and propositions
%\usepackage{amsthm}
% making logically defined graphics
%%%\usepackage{xypic}

% there are many more packages, add them here as you need them

% define commands here
\newcommand{\sijoitus}[2]%
{\operatornamewithlimits{\Big/}_{\!\!\!#1}^{\,#2}}
\begin{document}
We start from the first order differential equation
\begin{align}
(1\!-\!x^2)\frac{du}{dx}+2nxu \;=\; 0,
\end{align}
where one can \PMlinkname{separate the variables}{SeparationOfVariables} and then get the general solution
\begin{align}
u \;=\; C(1\!-\!x^2)^n.
\end{align}
Differentiating $n\!+\!1$ times the equation (1) it takes the form
$$(1\!-\!x^2)\frac{d^{n+2}u}{dx^{n+2}}-2x\frac{d^{n+1}u}{dx^{n+1}}+n(n+1)\frac{d^{n}u}{dx^{n}} \;=\; 0$$
or
\begin{align}
(1\!-\!x^2)\frac{d^2y}{dx^2}-2x\frac{dy}{dx}+n(n+1)y \;=\; 0
\end{align}
where
$$y \;=\; \frac{d^nu}{dx^n} \;=\; C\frac{d^n}{dx^n}(1\!-\!x^2)^n.$$
Especially, the particular solution
\begin{align}
y \;=\; P_n(x) \;:=\; \frac{1}{2^nn!}\frac{d^n}{dx^n}(1\!-\!x^2)^n,
\end{align}
which which is the Legendre polynomial of degree $n$, has been seen to satisfy the Legendre's differential equation (3).

The equality (4) is \PMlinkname{Rodrigues formula}{RodriguesFormula}.\, We use it to find the leading coefficient of $P_n(x)$ and to show the \PMlinkname{orthogonality}{OrthogonalPolynomials} of the Legendre polynomials $P_0(x),\,P_1(x),\;P_2(x),\,...$

\subsection{The coefficient of $x^n$}

By the binomial theorem, 
\begin{align*}
P_n(x) &\;=\; \frac{1}{2^nn!}\frac{d^n}{dx^n}\sum_{j=0}^n{n \choose j}x^{2(n-j)}(-1)^j\\
 &\;=\; \frac{1}{2^nn!}\sum_{j=0}^n{n \choose j}(2n\!-\!2j)(2n\!-\!2j\!-\!1)\cdots(2n\!-\!2j\!-\!n\!+\!1)x^{n-2j}(-1)^j.
\end{align*}
From the term with\, $j = 0$\, we get as the coefficient of $x^n$ the following:
\begin{align}
\frac{1}{2^nn!}{n\choose0}(2n)(2n\!-\!1)(2n\!-\!2)\cdots(2n\!-\!n\!+\!1)(-1)^0
\;=\; \frac{1}{2^nn!}\cdot\frac{(2n)!}{(2n\!-\!n)!} \;=\; \frac{(2n)!}{2^n(n!)^2}
\end{align}


\subsection{Orthogonality}

Let\, $f_m(x) := a_0\!+\!a_1x\!+\ldots+\!a_mx^m$\, be any polynomial of degree\, $m < n$.\, \PMlinkname{Integrating by parts}{IntegrationByParts} $m$ times we obtain
\begin{align*}
\int_{-1}^1f_m(x)P_n(x)\,dx & \;=\; \frac{1}{2^nn!}\int_{-1}^1f_m(x)\frac{d^n}{dx^n}(x^2\!-\!1)^n\,dx \\
& \;=\; \frac{1}{2^nn!}\sijoitus{-1}{\quad 1}\!f_m(x)\frac{d^{n-1}}{dx^{n-1}}(x^2\!-\!1)^n
-\frac{1}{2^nn!}\int_{-1}^1f'_m(x)\frac{d^{n-1}}{dx^{n-1}}(x^2\!-\!1)^n\,dx\\
& \cdots \qquad \cdots\\
& \;=\; (-1)^m\frac{a_mm!}{2^nn!}\int_{-1}^1\frac{d^{n-m}}{dx^{n-m}}(x^2\!-\!1)^n\,dx\\
& \;=\; (-1)^m\frac{a_mm!}{2^nn!}\sijoitus{-1}{\quad1}\!f_m(x)\frac{d^{n-m-1}}{dx^{n-m-1}}(x^2\!-\!1)^n \;=\; 0,
\end{align*}
since\, $x = \pm1$\, are zeros of the derivatives $\frac{d^{n-k}}{dx^{n-k}}(x^2\!-\!1)^n$.

If, on the other hand,\, $m = n$,\, the calculation gives firstly
\begin{align}
\int_{-1}^1f_n(x)P_n(x)\,dx \;=\; 2(-1)^n\frac{a_n}{2^n}\int_0^1(x^2\!-\!1)^n\,dx 
\;=\; 2(-1)^n\frac{a_n}{2^n}\cdot I_n,
\end{align}
where the integral $I_n$ is gotten from
$$I_n \;=\; \sijoitus{0}{\quad 1}\!x(x^2\!-\!1)^n-2n\int_0^1\!x^2(x^2\!-\!1)^{n-1}dx
\;=\; -2n\int_0^1\left[(x^2\!-\!1)^n+(x^2\!-\!1)^{n-1}\right]dx \;=\; -2nI_n-2nI_{n-1},$$
Thus we infer the recurrence relation
$$I_n \;=\; -\frac{2n}{2n\!+\!1}I_{n-1}.$$
Using this and\, $I_0 = 1$\, one easily arrives at
\begin{align}
I_n \;=\; (-1)^n\frac{2\cdot4\cdot6\cdots(2n)}{3\cdot5\cdot7\cdots(2n\!+\!1)}
\;=\; (-1)^n\frac{[2\cdot4\cdot6\cdots(2n)]^2}{(2n\!+\!1)!}
\;=\; (-1)^n\frac{2^{2n}(n!)^2}{(2n\!+\!1)!}.
\end{align}


If $f_n(x)$ also is a Legendre polynomial $P_n(x)$, we can in (6) by (5) put
$$a_n \;=\; \frac{(2n)!}{2^n(n!)^2}$$
and taking into account (7), too, (6) reads
$$\int_{-1}^1\left[P_n(x)\right]^2dx \;=\; 
\frac{(-1)^n}{2^{n-1}}\cdot\frac{(2n)!}{2^n(n!)^2}\cdot(-1)^n\frac{2^{2n}(n!)^2}{(2n\!+\!1)!}
\;=\; \frac{2}{2n\!+\!1}.$$

Our results imply the \PMlinkname{orthonormality}{Orthonormal} condition
\begin{align}
\int_{-1}^1\!P_m(x)P_n(x)\,dx \;=\; \frac{2}{2n\!+\!1}\delta_{mn},
\end{align}
where $\delta_{mn}$ is the Kronecker delta.

\begin{thebibliography}{9}
\bibitem{KK}{\sc K. Kurki-Suonio:} {\em Matemaattiset apuneuvot}.\, Limes r.y., Helsinki (1966).
\end{thebibliography}

%%%%%
%%%%%
\end{document}
