\documentclass[12pt]{article}
\usepackage{pmmeta}
\pmcanonicalname{GeneratingFunctionOfLegendrePolynomials}
\pmcreated{2015-03-08 20:29:06}
\pmmodified{2015-03-08 20:29:06}
\pmowner{pahio}{2872}
\pmmodifier{pahio}{2872}
\pmtitle{generating function of Legendre polynomials}
\pmrecord{17}{41793}
\pmprivacy{1}
\pmauthor{pahio}{2872}
\pmtype{Result}
\pmcomment{trigger rebuild}
\pmclassification{msc}{33B99}
\pmclassification{msc}{30D10}
\pmclassification{msc}{30B10}
%\pmkeywords{generating function}
%\pmkeywords{power series}
\pmrelated{GeneratingFunctionOfHermitePolynomials}
\pmrelated{GeneratingFunctionOfLaguerrePolynomials}
\pmrelated{VariantOfCauchyIntegralFormula}

\endmetadata

% this is the default PlanetMath preamble.  as your knowledge
% of TeX increases, you will probably want to edit this, but
% it should be fine as is for beginners.

% almost certainly you want these
\usepackage{amssymb}
\usepackage{amsmath}
\usepackage{amsfonts}

% used for TeXing text within eps files
%\usepackage{psfrag}
% need this for including graphics (\includegraphics)
%\usepackage{graphicx}
% for neatly defining theorems and propositions
 \usepackage{amsthm}
% making logically defined graphics
%%%\usepackage{xypic}

% there are many more packages, add them here as you need them

% define commands here

\theoremstyle{definition}
\newtheorem*{thmplain}{Theorem}

\begin{document}
For finding the generating function 
$$F(t) \;=\; \sum_{n=0}^\infty P_n(z)t^n$$
of the sequence of the Legendre polynomials\\
$P_0(z) \;=\; 1$ \\
$P_1(z) \;=\; z$ \\
$P_2(z) \;=\; \frac{1}{2}(3z^2\!-\!1)$ \\
$P_3(x) \;=\; \frac{1}{2}(5z^3\!-\!3z)$ \\
$P_4(z) \;=\; \frac{1}{8}(35z^4\!-\!30z^2\!+\!3)$ \\
$P_5(z) \;=\; \frac{1}{8}(63z^5\!-\!70z^3\!+\!15z)$\\
$\cdots \qquad\;\; \cdots$\\
we have to present $P_n(z)$ as the general coefficient of Taylor series in $t$, 
i.e. as the $n$th derivative of some $F(t)$ in the origin, divided by the factorial $n!$.\, The Cauchy integral formula offers the chance to implement that.

Starting from the \PMlinkid{Rodrigues formula}{11983} of Legendre polynomials, we may write
$$P_n(z) \;=\; \frac{1}{2^nn!}\frac{d^n}{dz^n}(z^2\!-\!1)^n \;=\; 
\frac{1}{2^nn!}\frac{n!}{2i\pi}\oint_c\frac{(\zeta^2\!-\!1)^n}{(\zeta\!-\!z)^{n+1}}d\zeta\;=\; 
\frac{1}{2i\pi}\oint_c\left(\frac{1}{2}\frac{\zeta^2\!-\!1}{\zeta\!-\!z}\right)^n\!\frac{d\zeta}{\zeta\!-\!z},$$
where the contour $c$ runs anticlockwise once around the point $z$.\, The change of variable
$$\frac{\zeta^2\!-\!1}{2(\zeta\!-\!z)} \;=\; \frac{1}{t}, 
\qquad d\zeta \;=\; \frac{zt\!-\!1\!-\!\sqrt{1\!-\!zt\!+\!t^2}}{t^2\sqrt{1\!-\!zt\!+\!t^2}}dt$$
gives
$$P_n(z) \;=\; -\frac{1}{2i\pi}\oint_{c'}\frac{dt}{t^nt\sqrt{1\!-\!zt\!+\!t^2}}$$
where $t$ must go round the origin clockwise, but in
$$P_n(z) \;=\; \frac{1}{n!}\cdot\frac{n!}{2i\pi}\oint_{c'}\frac{dt}{\sqrt{1\!-\!zt\!+\!t^2}\cdot(t\!-\!0)^{n+1}}$$
anticlockwise.\, This is, by Cauchy integral formula again, 
$$P_n(z) \;=\; \frac{1}{n!}\left[\frac{d^n}{dt^n}\frac{1}{\sqrt{1\!-\!zt\!+\!t^2}}\right]_{t=0}.$$
This means that
$$F(t) \;:=\; \frac{1}{\sqrt{1\!-\!zt\!+\!t^2}}$$
is the searched generating function of the Legendre polynomials:
$$\frac{1}{\sqrt{1\!-\!zt\!+\!t^2}} \;=\; P_0(z)+P_1(z)t+P_2(z)t^2+P_3(z)t^3+\ldots$$\\


Cf. the generating function of the Bessel functions.


%%%%%
%%%%%
\end{document}
