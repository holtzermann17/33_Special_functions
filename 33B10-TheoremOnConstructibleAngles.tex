\documentclass[12pt]{article}
\usepackage{pmmeta}
\pmcanonicalname{TheoremOnConstructibleAngles}
\pmcreated{2013-03-22 17:15:59}
\pmmodified{2013-03-22 17:15:59}
\pmowner{Wkbj79}{1863}
\pmmodifier{Wkbj79}{1863}
\pmtitle{theorem on constructible angles}
\pmrecord{13}{39605}
\pmprivacy{1}
\pmauthor{Wkbj79}{1863}
\pmtype{Theorem}
\pmcomment{trigger rebuild}
\pmclassification{msc}{33B10}
\pmclassification{msc}{51M15}
\pmclassification{msc}{12D15}
\pmrelated{ConstructibleNumbers}
\pmrelated{CompassAndStraightedgeConstruction}
\pmrelated{ConstructibleAnglesWithIntegerValuesInDegrees}
\pmrelated{ExactTrigonometryTables}
\pmrelated{ClassicalProblemsOfConstructibility}
\pmrelated{CriterionForConstructibilityOfRegularPolygon}

\endmetadata

\usepackage{amssymb}
\usepackage{amsmath}
\usepackage{amsfonts}
\usepackage{pstricks}
\usepackage{psfrag}
\usepackage{graphicx}
\usepackage{amsthm}
%%\usepackage{xypic}
\newtheorem{thm*}{Theorem}
\begin{document}
\PMlinkescapeword{constructible}
\PMlinkescapeword{even}
\PMlinkescapeword{label}
\PMlinkescapeword{leg}
\PMlinkescapeword{measure}
\PMlinkescapeword{measures}
\PMlinkescapeword{perpendicular}
\PMlinkescapeword{similar}
\PMlinkescapeword{vertex}

\begin{thm*}
Let $\theta \in \mathbb{R}$.  Then the following are equivalent:

\begin{enumerate}
\item An angle of \PMlinkname{measure}{AngleMeasure} $\theta$ is \PMlinkname{constructible}{Constructible2};
\item $\sin \theta$ is a constructible number;
\item $\cos \theta$ is a constructible number.
\end{enumerate}
\end{thm*}

\begin{proof}
First of all, due to periodicity, we can restrict our attention to the interval $0 \le \theta <2\pi$.  Even better, we can further restrict our attention to the interval $0 \le \theta \le \frac{\pi}{2}$ for the following reasons:

\begin{enumerate}
\item If an angle whose measure is $\theta$ is constructible, then so are angles whose measures are $\pi-\theta$, $\pi+\theta$, and $2\pi-\theta$;
\item If $x$ is a constructible number, then so is $|x|$.
\end{enumerate}

If $\theta \in \{0, \frac{\pi}{2} \}$, then clearly an angle of measure $\theta$ is constructible, and $\{\sin \theta, \cos \theta \}=\{0,1\}$.  Thus, \PMlinkname{equivalence}{Equivalent3} has been established in the case that $\theta \in \{0,\frac{\pi}{2}\}$.  Therefore, we can restrict our attention even further to the interval $0<\theta<\frac{\pi}{2}$.

Assume that an angle of measure $\theta$ is constructible.  Construct such an angle and mark off a line segment of length $1$ from the \PMlinkname{vertex}{Vertex5} of the angle.  Label the endpoint that is not the vertex of the angle as $A$.

\begin{center}
\begin{pspicture}(-1,-1)(2,3)
\rput[l](-0.1,0){.}
\rput[r](2,0){.}
\rput[a](2,3.464){.}
\psline{->}(0,0)(2,0)
\psline{->}(0,0)(2,3.464)
\psarc(0,0){0.3}{0}{60}
\rput[r](0.5,0.3){$\theta$}
\psarc[linecolor=blue](0,0){3}{50}{70}
\psdots(0,0)(1.5,2.598)
\rput[b](1.2,2.3){$A$}
\end{pspicture}
\end{center}

Drop the perpendicular from $A$ to the other ray of the angle.  Since the legs of the triangle are of lengths $\sin \theta$ and $\cos \theta$, both of these are constructible numbers.

\begin{center}
\begin{pspicture}(-1,-1)(2,4)
\rput[l](-0.1,0){.}
\rput[r](2,0){.}
\rput[a](2,3.464){.}
\rput[b](1.5,-1){.}
\psline{->}(0,0)(2,0)
\psline{->}(0,0)(2,3.464)
\psarc(0,0){0.3}{0}{60}
\rput[r](0.5,0.3){$\theta$}
\psarc(0,0){3}{50}{70}
\psline[linecolor=blue]{<->}(1.5,3)(1.5,-1)
\psdots(0,0)(1.5,2.598)(1.5,0)
\rput[b](1.2,2.3){$A$}
\rput[a](0.7,-0.3){$\cos \theta$}
\rput[l](1.7,1.3){$\sin \theta$}
\end{pspicture}
\end{center}

Now assume that $\sin \theta$ is a constructible number.  At one endpoint of a line segment of length $\sin \theta$, erect the perpendicular to the line segment.

\begin{center}
\begin{pspicture}(-1,-1)(3,2)
\rput[l](-0.1,0){.}
\rput[r](3,0){.}
\rput[a](2.598,2){.}
\rput[b](2.598,-1){.}
\psline{->}(0,0)(3,0)
\psline[linecolor=blue]{<->}(2.598,-1)(2.598,2)
\psdots(0,0)(2.598,0)
\end{pspicture}
\end{center}

From the other endpoint of the given line segment, draw an arc of a circle with radius $1$ so that it intersects the erected perpendicular.  Label this point of intersection as $A$.  Connect $A$ to the endpoint of the line segment which was used to draw the arc.  Then an angle of measure $\theta$ and a line segment of length $\cos \theta$ have been constructed.

\begin{center}
\begin{pspicture}(-1,-1)(3,2)
\rput[l](-0.1,0){.}
\rput[r](3,0){.}
\rput[a](2.598,2){.}
\rput[b](2.598,-1){.}
\psline{->}(0,0)(3,0)
\psline{<->}(2.598,-1)(2.598,2)
\psarc[linecolor=blue](0,0){3}{20}{40}
\psline[linecolor=blue](0,0)(2.598,1.5)
\psarc(2.598,1.5){0.3}{210}{270}
\rput[a](2.4,1){$\theta$}
\rput[l](2.8,0.8){$\cos \theta$}
\psdots(0,0)(2.598,0)(2.598,1.5)
\rput[b](2.3,1.5){$A$}
\end{pspicture}
\end{center}

A similar procedure can be used given that $\cos \theta$ is a constructible number to prove the other two statements.
\end{proof}

Note that, if $\cos \theta \neq 0$, then any of the three statements thus implies that $\tan \theta$ is a constructible number.  Moreover, if $\tan \theta$ is constructible, then a right triangle having a leg of length $1$ and another leg of length $\tan \theta$ is constructible, which implies that the three listed conditions are true.
%%%%%
%%%%%
\end{document}
